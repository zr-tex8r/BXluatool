% 文字コードは UTF-8
% lualatex で組版する(LuaTeX-ja 使用)
\documentclass[a4paper]{ltjsarticle}
\usepackage{shortvrb}
\newcommand{\PkgVersion}{0.2c}
\newcommand{\PkgDate}{2012/06/05}
\MakeShortVerb{\|}
\newcommand{\Pkg}[1]{\textsf{#1}}
\newcommand{\Meta}[1]{$\langle$\mbox{}#1\mbox{}$\rangle$}
\newcommand{\Note}{\par\noindent ※}
\newcommand{\Means}{~:\quad}
\newcommand{\Expl}{\par\noindent 例:\ }
\newcommand{\Jemph}{\textbf}
\newcommand{\Typ}[1]{\textsl{#1}}
\newcommand{\Var}[1]{\textit{#1}}
\newcommand{\NumberT}{\Typ{number}}
\newcommand{\NumVar}{\NumberT\ \Var}
\newcommand{\StringT}{\Typ{string}}
\newcommand{\StrVar}{\StringT\ \Var}
\newcommand{\BooleanT}{\Typ{boolean}}
\newcommand{\BoolVar}{\BooleanT\ \Var}
\newcommand{\TableT}{\Typ{table}}
\newcommand{\TabVar}{\TableT\ \Var}
\newcommand{\GluespecT}{\Typ{gluespec}}
\newcommand{\GlueVar}{\GluespecT\ \Var}
%-----------------------------------------------------------
\begin{document}
\title{\Pkg{bxluafacade} パッケージ(v\PkgVersion)}
\author{八登崇之\ (Takayuki YATO; aka.~``ZR'')}
\date{\PkgDate}
\maketitle

%===========================================================
\section{概要}
\label{sec:overview}

本パッケージは以下の機能を提供する。
\begin{itemize}
\item {\LaTeX}上の多少複雑な処理をLuaプログラムを用いて
  解決する際に、必要となる機能を用意する。
\item 主に、{\LaTeX}とLuaの間のインタフェースに関わる
  機能である。
\item 従って、本パッケージは{\LaTeX}とLuaの知識さえあれば、
  {\TeX}の知識がなくとも使えるようにすることを目指す。
\end{itemize}

\paragraph{前提環境}
\begin{itemize}
\item {\TeX}フォーマット: Lua{\LaTeX}
\end{itemize}

\paragraph{依存パッケージ}
\begin{itemize}
\item \Pkg{etoolbox} パッケージ
\item \Pkg{luatexbase} パッケージ
\end{itemize}

%===========================================================
\section{パッケージの読込}
\label{sec:load}

|\usepackage| で読み込む。
オプションはない。
\begin{verbatim}
\usepackage{bxluafacade}
\end{verbatim}

%===========================================================
\section{基本的な機能({\LaTeX}側)}
\label{sec:basic-latex}

本節の説明中の「例」においては以下に示す定義が
行われていることを前提とする。

\begin{quote}\small\begin{verbatim}
\newcommand*{\macroA}{\macroB}
\newcommand*{\macroB}{I'd}
\newcommand*{\Answer}{42}
\newcounter{answer} \setcounter{answer}{42}
\newlength{\sample} \setlength{\sample}{1ptplus1pt}
\end{verbatim}\end{quote}

\begin{itemize}
\item |\luaescape{|\Meta{テキスト}|}|\Means
  |\luastring| と同様に、
  引数(完全展開する)を文字列化したものに
  Lua文字列リテラルのためのエスケープを施すが、
  |" "| では囲わない。
  \Expl |s = "\luaescape{\macroA}"|
        →(展開:|s = "I\'d"|)
        → |s| は ``|I'd|''
\item |\luaescapeO{|\Meta{テキスト}|}|\Means
  |\luastringO| と同様に、
  引数(一回展開する)を文字列化したものに
  Lua文字列リテラルのためのエスケープを施すが、
  |" "| では囲わない。
  \Expl |s = "\luaescapeO{\macroA}"|
        →(展開:|s = "\\macroB "|)
        → |s| は ``|\macroB |''
\item |\luaescapeN{|\Meta{テキスト}|}|\Means
  |\luastringN| と同様に、
  引数(展開しない)を文字列化したものに
  Lua文字列リテラルのためのエスケープを施すが、
  |" "| では囲わない。
  \Expl |s = "\luaescapeN{\macroA}"|
        →(展開:|s = "\\macroA "|)
        → |s| は ``|\macroA |''
\item |\luanumber{|\Meta{数値}|}|\Means
  引数(完全展開される)をLuaで数値として扱う式に変換する。
  引数が直接数字列で書かれた数値でない場合はnilと解釈される。
  \Expl |n = \luanumber{\Answer}|
        →(展開:|n = tonumber("42")|)
        → |n| は42
  \Expl |n = \luanumber{6*9}|
        →(展開:|n = tonumber("6*9")|)
        → |n| はnil
  \Expl |n = \luanumber{\value{answer}}|
        →(展開:|n = tonumber("\c@answer ")|)
        → |n| はnil
\item |\luanumberN{|\Meta{数値}|}|\Means
  引数(展開しない)をLuaで数値として扱う式に展開する。
  引数が直接数字列で書かれた数値でない場合はnilと解釈される。
  \Expl |n = \luanumberN{\Answer}|
        →(展開:|n = tonumber("\Answer ")|)
        → |n| はnil
  \Expl |n = \luanumberN{3.14159}|
        →(展開:|n = tonumber("3.14159")|)
        → |n| は3.14159
\item |\luacounterval{|\Meta{カウンタ名}|}|\Means
  {\LaTeX}カウンタの現在の値をLuaで数値として扱う式に展開する。
  \Expl |n = \luacounterval{answer}|
        →(展開:|n = (42)|)
        → |n| は42
\item |\lualengthval{\長さ変数名}|\Means
  長さ変数の現在の値をLuaでグル―値(gluespec)
  として扱う式に展開する。
  \Expl |g = \lualengthval{\sample}|
        →(展開:|g = bxlt.to_skip("1pt plus 1pt")|)
        → |g| は「1pt plus 1pt」を表すgluespec
\end{itemize}

%===========================================================
\section{基本的な機能(Lua側)}
\label{sec:basic-lua}

\begin{itemize}
\item |bxlt.print(|[\NumVar{c}|,|] \StrVar{s}|,| |...)|\Means
  引数の各文字列を改行文字(|'\n'|)毎に分割して\ 
  |tex.print(|[\Var{c}|,|] |...)| に渡す
  (文字列の末尾の改行は無視する)。 
  結果的に、文字列の中の各行が{\TeX}の1つの入力行と扱われる。
  \Expl 例えば |tex.print("A%B\nC")|
  は「A」しか出力しないが、
  |bxlt.print("A%B\nC")|
  は |tex.print("A%B", "C")|
  と同じで「A C」と出力する。
\item |bxlt.printf(|[\NumVar{c}|,|] \StrVar{f}|,| |...)|\Means
  書式付で{\TeX}にテキスト出力(print版)。
  |bxlt.print(|[\Var{c}|,|] |string.format(|\Var{f}|,| |...)|
  と同じ。
  \Expl |bxlt.printf("%02X", 42)|
  は「|2A|」を{\TeX}に書き出す。
\item |bxlt.writef(|[\NumVar{c}|,|] \StrVar{f}|,| |...)|\Means
  書式付で{\TeX}に文字列出力(write版)。
  |tex.write(|[\Var{c}|,|] |string.format(|\Var{f}|,| |...)|
  と同じ。
  \Expl |bxlt.writef("[%s]", "C#")|
  はverbatimに(カテゴリコード12で)|[C#]| を{\TeX}に書き出す。
\item |bxlt.sprintf(|[\NumVar{c}|,|] \StrVar{f}|,| |...)|\Means
  書式付で{\TeX}にテキスト出力(sprint版)。
  \footnote{C言語の \texttt{sprintf()} に相当するのは\ 
    \texttt{string.format()}。}
  |bxlt.sprint(|[\Var{c}|,|] |string.format(|\Var{f}|,| |...)|
  と同じ。
\item (1) \NumVar{sp} |=| |bxlt.to_dimen(nil)|\\
  (2) \NumVar{sp} |=| |bxlt.to_dimen(|\NumVar{d}|)|\\
  (3) \NumVar{sp} |=| |bxlt.to_dimen(|\StrVar{d}|)|\Means
  長さの表現を、sp単位の整数値に変換する。
  (1)は0を返す。
  (2)は\Var{d}をそのまま返す。
  (3)は\Var{d}を{\TeX}での長さ表記(単位付き数値)として
  解釈してsp単位の整数値を返す。
  \Expl |sp = bxlt.to_dimen(26214)| → |sp| は26214。
  \Expl |sp = bxlt.to_dimen("1cm")| → |sp| は1864679。
\item (1) \GlueVar{gs} |=| |bxlt.to_skip(nil)|\\
  (2) \GlueVar{gs} |=| |bxlt.to_skip(|\NumVar{s}|)|\\
  (3) \GlueVar{gs} |=| |bxlt.to_skip(|\GlueVar{s}|)|\\
  (4) \GlueVar{gs} |=| |bxlt.to_skip(|\StrVar{s}|)|\Means
  グル―値(伸縮する長さ)の表現を、gluespecに変換する。
  (1)は「0pt」のgluespecを返す。
  (2)は「\Var{s}\,sp」のgluespecをそのまま返す。
  (3)は\Var{s}自身を返す。
  (4)は\Var{s}を{\TeX}でのグル―値表記として
  解釈して結果のgluespecを返す。
  \Expl |sp = bxlt.to_skip(16384)| → |sp| は「0.25pt」のgluespec。
  \Expl |sp = bxlt.to_dimen("1pt minus .1pt")| → |sp| は「1pt minus 0.1pt」のgluespec。
\end{itemize}

%===========================================================
\section{Quick-escape}
\label{sec:quick}
\newcommand*{\XOth}{\underline{\texttt{*}}}
\newcommand*{\XLtr}{\underline{\texttt{a}}}

{\LaTeX}コード中で |\directlua| でLuaコードを記述する場合、
{\LaTeX}の特殊文字(|\| や |%|)
はその特殊な意味を保つので、
そのままでは記述することができない。
その場合はできるだけ別ファイル(Luaソースファイル)に
Luaコードを移動するか、あるいは、
\Pkg{luacode}パッケージが提供する |luacode*| 関数を
利用することが通常の方法である。

しかし、場合によっては |\directlua| の中に
特殊文字を書けた方が便利であると考える人もいるであろう。
Quick-escapeはそのような要求を満たすための機能である。

Quick-escapeを有効にすると、Backtick記号〈|`|〉と
その後に続く1つのトークンについて、
以下の変換が行われる。
({\XLtr} は任意のASCII英字、{\XOth} は任意のASCII非英字を表す。)
\begin{itemize}
\item |`\|{\XOth} は {\XOth} に展開される。
  よって、任意の{\LaTeX}特殊文字(全てASCII非英字である)
  について、前に |`\| を前置することでLuaコードに含めることができる。
  \Expl |("[`\%3d]`\\n", `\#str)| → |("[%3d]\n", #str)|
\item |`|{\XLtr} は |`|{\XLtr} に展開される。
  \Expl |p("`Yes!'")| → |p("`Yes!'")|
\item |`|{\XOth} の組み合わせのうち一部は以下のように変換される。
  \begin{center}
  \newcommand{\?}[1]{\texttt{\symbol{`#1}}}
  \begin{tabular}{*8{c}}
    \?\+ & \?\& & \?\| & \?\- & \?\( & \?\) & \?\< & \?\> \\
    ↓ & ↓ & ↓ & ↓ & ↓ & ↓ & ↓ & ↓ \\
    \?\# & \?\% & \?\\ & \?\~ & \?\{ & \?\} & \?\^ & \?\\\?\\ \\
  \end{tabular}
  \end{center}
  例: |p("`>def`-`+1`(`)")| → |p("\\def~#1{}")|
\item 以上に挙げたもの以外の組み合わせについては未定義である。
\end{itemize}

Quick-escapeは既定では無効になっている。
以下の命令を用いて有効/無効を切り替える。
\begin{itemize}
\item |\usequickescape|\Means
  Quick-escapeを有効にする。
  \Note |`| のカテゴリコードを(ローカルに)13にする。
  初回の実行では以下の初期化処理がグローバルに行われる:
  アクティブな |`| を |\qesc| と等価にし、
  |`| のカテゴリコードを12にする。
\item |\nousequickescape|\Means
  Quick-escapeを無効にする。
  \Note |`| のカテゴリコードを(ローカルに)12にする。
\item |\finishquickescape|\Means
  Quick-escapeの使用を終了。
  \Note アクティブな |`| の意味と |`| のカテゴリコードを
  初期化処理の直前の状態にグローバルに戻す。
  これを実行してしまうと、
  それまでquick-escapeを用いて定義していた命令(マクロ)
  が正しく動作しなくなることに注意せよ。
\item |\qesc|\Means
  Quick-escapeの処理の実体。
  Quick-escapeが無効の場合でも、|`| の代わりに |\qesc|
  を用いてquick-escapeが実行できる。
\end{itemize}

\end{document}
