% 文字コードは UTF-8
% lualatex で組版する(LuaTeX-ja 使用)
\documentclass[a4paper]{ltjsarticle}
\usepackage{shortvrb}
\newcommand{\PkgVersion}{0.2}
\newcommand{\PkgDate}{2012/06/05}
\MakeShortVerb{\|}
\newcommand{\Pkg}[1]{\textsf{#1}}
\newcommand{\Meta}[1]{$\langle$\mbox{}#1\mbox{}$\rangle$}
\newcommand{\Note}{\par\noindent ※}
\newcommand{\Means}{~:\quad}
\newcommand{\Expl}{\par\noindent 例:\ }
\newcommand{\Jemph}{\textbf}
\newcommand{\Typ}[1]{\textsl{#1}}
\newcommand{\Var}[1]{\textit{#1}}
\newcommand{\NumberT}{\Typ{number}}
\newcommand{\NumVar}{\NumberT\ \Var}
\newcommand{\StringT}{\Typ{string}}
\newcommand{\StrVar}{\StringT\ \Var}
\newcommand{\BooleanT}{\Typ{boolean}}
\newcommand{\BoolVar}{\BooleanT\ \Var}
\newcommand{\TableT}{\Typ{table}}
\newcommand{\TabVar}{\TableT\ \Var}
\newcommand{\GluespecT}{\Typ{gluespec}}
\newcommand{\GlueVar}{\GluespecT\ \Var}
%-----------------------------------------------------------
\begin{document}
\title{\Pkg{bxluatangle} パッケージ(v\PkgVersion)}
\author{八登崇之\ (Takayuki YATO; aka.~``ZR'')}
\date{\PkgDate}
\maketitle

%===========================================================
\section{概要}
\label{sec:overview}

本パッケージは以下の機能を提供する。
\begin{itemize}
\item Luaの実行コンテキストの途中で、
  それまで |tex.print()| 等で{\TeX}側に書き出したコードの
  実行を完了させる。
\end{itemize}

\paragraph{前提環境}
\begin{itemize}
\item {\TeX}フォーマット: Lua{\LaTeX}
\end{itemize}

\paragraph{依存パッケージ}
\begin{itemize}
\item \Pkg{luatexbase} パッケージ
\item \Pkg{luacode} パッケージ
\end{itemize}

%===========================================================
\section{パッケージの読込}
\label{sec:load}

|\usepackage| で読み込む。
オプションはない。
\begin{verbatim}
\usepackage{bxluatangle}
\end{verbatim}

%===========================================================
\section{動機}
\label{sec:motivation}

例えば、あるLua関数の実行の中で、
与えられたテキストを組版した時の幅と
与えられた長さとの大小により
処理を分ける必要があったとする。
この状況を抽出して、次のようなコードを考えてみる。
(\Pkg{luacode}および\Pkg{bxluafacade}の読込を前提とする。)

\begin{quote}\small\begin{verbatim}
\setlength{\reswidth} % 長さ変数
\newcommand{\widthtest}[2]{\directlua{
  width_test(\luastringN{#1}, \luastringN{#2})
}}
\begin{luacode*}
-- text: テキスト
-- thr: 長さ
function width_test(text, thr)
  local thr_val = bxlt.to_dimen(thr)
  -- TeX の \settowidth を用いて幅を得る
  tex.print("\\settowidth{\reswidth}{"..text.."}")
  -- 【ここで一旦TeXの実行を完了させたい】
  if bxlt.length.reswidth.width <= thr_val then
    print("short!")
  else
    pritn("long!")
  end
end
\end{luacode*}
\end{verbatim}\end{quote}

上掲のコードはこのままでは正常に動作しない。
|tex.print()| で |\settowidth| 命令を書き出しているが、
実際にそれが実行されるのは、|width_test()| を呼び出している\ 
|\directlua| が終了してからである。

従って、通常は、こういう処理をしようと思うと、
「{\TeX}の実行を完了させる必要がある箇所」
で一旦Luaの処理を中断できるようにプログラムを工夫する
必要が生じる。
仮にその箇所が深いループの中だったりすると
その実現は困難を極めることになる。

本パッケージを用いると、
今の要望が次のようにして実現できる。

\begin{quote}\small\begin{verbatim}
\setlength{\reswidth}
\newcommand{\widthtest}[2]{\tangleluaexec{% tangle付で実行
  width_test(\luastringN{#1}, \luastringN{#2})
}}
\begin{luacode*}
function width_test(text, thr)
  local thr_val = bxlt.to_dimen(thr)
  tex.print("\\settowidth{\reswidth}{"..text.."}")
  bxlt.run_tex()
  if bxlt.length.reswidth.width <= thr_val then
    print("short!")
  else
    pritn("long!")
  end
end
\end{luacode*}
\end{verbatim}\end{quote}

すなわち、Luaの実行を開始するときに、
通常の |\luadirect| でなく |\tangleluaexec| を用いる
(これを「tangle付でLuaを実行する」と呼ぶ)。
「tangle付」のLuaの実行の中では、|bxlt.run_tex()|
を任意の箇所で呼び出して、
それまでに |tex.print()| で書き出したコードを
実行させることができるようになる。

%===========================================================
\section{基本的な機能}
\label{sec:basic}
\newcommand*\Cs[1]{\texttt{\symbol{`\\}#1}}

{\LaTeX}側の機能。

\begin{itemize}
\item |\tangleluaexec{|\Meta{Luaコード}|}|\Means
  \Pkg{luaexec}の |\luaexec| と同じコード記述規定で、
  tangle付でLuaコードを実行する。
  元々 |\directlua| で呼び出していたコードをtangle付にする
  場合にもこれを用いる。
  \footnote{ただし、\Cs{tangleluaexec} は \Cs{directlua}
    と異なり展開可能ではない。
    現状の実装方法では、tangle付の実行は展開可能にはできない。}
\item |tangleluacode| 環境\Means
  \Pkg{luaexec}の |luacode| と同じコード記述規定で、
  tangle付でLuaコードを実行する。
\item |tangleluacode*| 環境\Means
  \Pkg{luaexec}の |luacode*| と同じコード記述規定で、
  tangle付でLuaコードを実行する。
\end{itemize}

Lua側の機能。

\begin{itemize}
\item |bxlt.run_tex()|\Means
  それまでに書き出された{\TeX}コードの実行を完了させる。
\end{itemize}

\end{document}
